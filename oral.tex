\documentclass[landscape,dvips,a4]{seminar}
\usepackage[english]{babel}
\usepackage[slides,chinese]{research11}
%\usepackage[retainorgcmds]{IEEEtrantools}
\renewcommand{\myfooter}{}
\begin{document}
\sffamily

%%%slide1
\begin{slide}
  \begin{center}
    %\mbox{}    
    \vspace{-0.3cm}
    \hspace{-0.3cm}
    \begin{minipage}{0.18\textwidth}
      \includegraphics[height=1.75cm]{NCTU_Logo.eps}
    \end{minipage} 
    \begin{minipage}{0.28\textwidth}
      \flushleft 
      National Chiao Tung University Taiwan

    \end{minipage} 
    \hfill
    \begin{minipage}{0.25\textwidth}
      \flushright
      Institute of Communications Engineering
    \end{minipage} 
    \begin{minipage}{0.23\textwidth}
    \hfill 
      \includegraphics[height=1.4cm]{cm_logo.eps}
    \end{minipage} 


    \vs{2}
    Date : \today\\[5mm]
    \LARGE
    \textbf{SIMO Fading Channel\\ with Memory and Feedback}
    
    \normalsize
    \vs{2}
    \cjktexta{郭沅竹}

    \vs{3}
     Advisor: Prof. Stefan M. Moser
    \vs{1}
  \end{center}
  \setlength{\unitlength}{0.1mm}
  \begin{picture}(0,0)
  \end{picture}
\end{slide}
% The asymptotic channel capacity of a general noncoherent regular
%   single-input multiple-output fading channel with memory and with
%   feedback is investigated. 
% Talk about outline
% Before we have fading with memory, now we add feedback

%%%slide2 
\begin{slide}
  \begin{center}
    \textbf{\Large Motivation}
  \end{center}
 \vspace{3mm}
\begin{figure}[htbp]
   \centering
   \includegraphics[width=0.5\textwidth]{SIMO2.eps}
   \caption{SIMO}
   \label{fig:SIMO}
 \end{figure}
   \begin{itemize}
     \item SIMO: {\b{single-input multiple-output}}
   \end{itemize}
\vs{2}
\end{slide}
% graph: mobile v.s. basestation
% Fading channel missing
% Wireless
% multipath, so fading
%% "Now" we have basic picture, "next" we give more detail about model


%%%slide2 
\begin{slide}
  \begin{center}
    \textbf{\Large Motivation}
  \end{center}
 \vspace{3mm}
\begin{figure}[htbp]
   \centering
   \includegraphics[width=0.5\textwidth]{SIMO3.eps}
   \caption{SIMO}
   \label{fig:SIMO}
 \end{figure}

\vs{2}
\end{slide}



%%%slide4 
\begin{slide}
  \begin{center}
    \textbf{\Large Channel Model}
  \end{center}
 \vspace{3mm}
\begin{figure*}[t]
\centering
\psfrag{M}{$M$}
\psfrag{Mh}{$\hat{M}$}
\psfrag{trans}{Tx}
\psfrag{recei}{Rx}
\psfrag{delay}[cc][cc]{delay}
\psfrag{X}{$X_k$}
\psfrag{Y1}{{$Y_k^{(1)}$}}
\psfrag{Yn}{{$Y_k^{(\nr)}$}}
\psfrag{H1}{{$H_k^{(1)}$}}
\psfrag{Hn}{{$H_k^{(\nr)}$}}
\psfrag{Z1}{{$Z_k^{(1)}$}}
\psfrag{Zn}{{$Z_k^{(\nr)}$}}
\psfrag{Fk}{$\vect{F}_k$}
\psfrag{Fk1}{$\vect{F}_{k-1}$}
\includegraphics[width=0.84\textwidth]{simofeedback.eps}
\caption{Channel Model}
\label{fig:simofeedback}
\end{figure*}

   \begin{itemize}
   \item Noncoherent
   \item {\b{Feedback}}
     \begin{itemize}
  \item noisefree
       \item causal: $\vect{F}_k=\vect{Y}_1^{k-1}$
     \end{itemize}
   \end{itemize}
\vs{2}
\end{slide}
% k: time step
% Two kind of noise: multiplicative noise H and additive noise N
% Noncoherent: we allow measuring channel, but with cost, ex: need
% more power to do that
% Noiseless, which means the link has infinite capacity, and allows
% the receiver to send everything it knows back to the transmitter. 
% noiseless feedback is unrealistic but it is an upper bound to all
% the situations! we know it is too optimistic
% feedback can only help
%% Mathematical model, more detail about H and Z, how this two noise connected with our signal

%%%slide5 
\begin{slide}
  \begin{center}
    \textbf{\Large Channel Model (Cont')}
  \end{center}
 \vspace{3mm}
 \begin{equation*}
   \vect{Y}_k={\b{\vect{H}_k}}x_k+{\b{\vect{Z}_k}}
 \end{equation*}
   \begin{itemize}
   \item $\{\vect{Z}_k\}$ IID $\sim \NormalC{\vect{0}}{\sigma^2
    \mat{I}_{\nr}}$
      \item $\{\vect{H}_k\} \indep \{\vect{Z}_k\}$
    \item $\{\vect{H}_k\}$: {\r{very general}}
      \begin{itemize}
      \item stationary, ergodic, finite energy: $\E{\|\vect{H}_k\|^2} < \infty$
          \item with memory          
\item {\b finite} differential entropy rate ({\b regular}): $h(\{\vect{H}_k\}) > -\infty$
        \end{itemize} 
        \item Power constraints: {\b average} or {\b peak} for power $\mathcal{E}$
\end{itemize}
\vs{2}
\end{slide}
% H is general, ergodic (without this, we cannot compute capacity)
% Memory: current rely on the past
% h(H)>-infty: strength of the memory, keep certain randomness, so even if we have infinite
% past, we cannot predict the whole channel in the future
% Why h(H)>-infty? there are two (regular and nonregular), both are
% decent assumption, we focus on the regular one
% capacity is a function of E and we are interested in E to infinity
%% We have specify the setup, now we will review the previous result;

%%%slide6 
\begin{slide}
  \begin{center}
    \textbf{\Large Asymptotic Capacity {\b{without}} Feedback}
  \end{center}
 \vspace{3mm}
\begin{equation*}
  \const{C}(\mathcal{E}) = \log(1+\log(1+\mathcal{E})) + {\b{\chi}}(\{\vect{H}_k\}) + o(1)
\end{equation*}
\begin{itemize}
\item ${\b{\chi}}$: \emph{fading number} %importance of the fading number
\end{itemize}
\vs{2}
\end{slide}
% C is a function of power
% C=... asymptotically
% Has been shown before that loglog is extremely slow and is affected by fading process
% \chi: not depend on power but on H's assumption
%% loglog only depend on power, not on N_R, {H}
%% loglog is very bad, so we want to know the threshold
%% what we are really interested is fading

%%%slide7
\begin{slide}
  \begin{center}
    \textbf{\Large Fading Number}
  \end{center}
 \vspace{3mm}
 \begin{figure}[htbp]
    \centering
 \psfrag{log-regime}{log-regime}
 \psfrag{loglog-regime}{loglog-regime}
 \psfrag{threshold}{threshold} 
\psfrag{SNR}{SNR}
 \psfrag{f}{$\chi$}
 \psfrag{C}{$\const{C}$}
\psfrag{o}{$o(1)$}
\psfrag{d}{dominating}
\psfrag{x}{$\chi$}
\psfrag{s}{$\log(1+\log(1+\textnormal{SNR}))$}
 \includegraphics[width=0.8\textwidth]{fadingNumber.eps}
    \caption{Capacity of a typical regular fading channel}
    \label{fig:fadingNumber}
  \end{figure}
\vs{2}
\end{slide}
% for knowing threshold, we need to know the asyptotic capacity
% o(1) is already is near the threshold, and loglog is not big yet
% Importance of fading number
% Low: normal log; high: extremely slow, emphasize by example
% Threshold: sensitive to assumption, so x-axis to y-axis
% Three region: low, threshold, high
%% real math form of the fading number

%%%slide8 
\begin{slide}\addtocounter{slide}{-1}
\k
  \begin{center}
    \textbf{\Large Fading Number {\b{without}}
      Feedback} 
  \end{center}
 \vspace{3mm}

\begin{equation*}
  \const{C}(\mathcal{E}) = \log(1+\log(1+\mathcal{E})) + {\b{\chi}}(\{\vect{H}_k\}) + o(1)
\end{equation*}
\g
\begin{itemize}
\item $\chi$: \emph{fading number} %importance of the fading number
\begin{IEEEeqnarray*}{rCl}
  \chi(\{\vect{H}_k\}) 
  & = & h_{\lambda}\left( \hvect{H}_0 e^{\ii\Theta_0}
    \middle| \big\{\hvect{H}_{\ell} 
    e^{\ii\Theta_{\ell}}\big\}_{\ell=-\infty}^{-1} \right) 
  - \log 2 \nonumber\\
  && +\: \nr \E{\log \|\vect{H}_0\|^2} - h\left( \vect{H}_0 \middle|
    \vect{H}_{-\infty}^{-1} \right)
\end{IEEEeqnarray*}
\begin{itemize}
\item $\{\Theta_k\}$ IID $\sim\Uniform{(-\pi, \pi]}$, $\indep \{\vect{H}_k\}$
\item $\hvect{H}_k \eqdef \frac{\vect{H}_k}{\|\vect{H}_k\|}$
\item $h_{\lambda}(\hvect{V}) \eqdef \Em{\big}{{-}{\log
    p_{\hvect{V}}^{\lambda}(\hvect{V})}}$ 
\end{itemize}
\end{itemize}
%\end{overlay}
\vs{2}
\end{slide}
% \chi: a constant, depends only of {H_k}; E[]; h[]; H_hat
% \Theta: it turns out that it is optimal by choosing it to be
% circularly symmetric because it can be put most information
% H_hat: diretion. we try to find the mag, direc., phase. for input,
% we choose mag(power) to be infinite and phase to be circularly symmetric
% h_\lambda: entropy rate for a unit vector
% Memory with feedback usually increase capacity
%% Comes to our main result, feedback should help: bra bra bra
%% bad news is 


%%%slide7 
\begin{slide}\addtocounter{slide}{-1}
\k
  \begin{center}
    \textbf{\Large Fading Number {\b{without}}
      Feedback} 
  \end{center}
 \vspace{3mm}
\g
\begin{equation*}
  \const{C}(\mathcal{E}) = \log(1+\log(1+\mathcal{E})) + \chi(\{\vect{H}_k\}) + o(1)
\end{equation*}
\k
\begin{itemize}
\item ${\b{\chi}}$: \emph{fading number} %importance of the fading number
\begin{IEEEeqnarray*}{rCl}
  {\b{\chi}}(\{\vect{H}_k\}) 
  & = & h_{\lambda}\left( \hvect{H}_0 e^{\ii\Theta_0}
    \middle| \big\{\hvect{H}_{\ell} 
    e^{\ii\Theta_{\ell}}\big\}_{\ell=-\infty}^{-1} \right) 
  - \log 2 \nonumber\\
  && +\: \nr \E{\log \|\vect{H}_0\|^2} - h\left( \vect{H}_0 \middle|
    \vect{H}_{-\infty}^{-1} \right)
\end{IEEEeqnarray*}
\begin{itemize}
\item $\{\Theta_k\}$ IID $\sim\Uniform{(-\pi, \pi]}$, $\indep \{\vect{H}_k\}$
\item $\hvect{H}_k \eqdef \frac{\vect{H}_k}{\|\vect{H}_k\|}$
\item $h_{\lambda}(\hvect{V}) \eqdef \Em{\big}{{-}{\log
    p_{\hvect{V}}^{\lambda}(\hvect{V})}}$ 
\end{itemize}
\end{itemize}
%\end{overlay}
\vs{2}
\end{slide}


%%%slide7 
\begin{slide}
  \begin{center}
    \textbf{\Large Asymptotic Capacity {\b{with}} Feedback}
  \end{center}
 \vspace{3mm}
\begin{whiteboxtheorem}%[SIMO Fading Number with Feedback]
  \label{thm:feedbackfadingnumber}
  \begin{equation*}
     \const{C}_{\textnormal{FB}}(\mathcal{E}) 
     = \log(1+\log(1+\mathcal{E})) + \chi_{\textnormal{FB}}(\{\vect{H}_k\}) + o(1)
  \end{equation*}

  \vspace{-2.5mm}
\end{whiteboxtheorem}
\vs{2}
\end{slide}
  % Let a general SIMO fading channel with memory be defined as in
  % \eqref{eq:2} and consider a noiseless causal feedback link as
  % described in \eqref{eq:6} (see Figure~\ref{fig:fig1}). Then the
  % asymptotic channel capacity under either an average-power constraint
  % \eqref{eq:av} or a peak-power constraint \eqref{eq:pp} is identical
  % to the asymptotic channel capacity for the channel without feedback:
% the asymptotic capacity grows double-logarithmically in the
% power and that the second term in the asymptotic expansion, the
% \emph{fading number}, is unchanged with respect to the same channel
% without feedback.
% first term loglog does not change

\begin{slide}
  \begin{center}
    \textbf{\Large Fading Number {\b{with}} Feedback}
  \end{center}
 \vspace{3mm}
\begin{whiteboxtheorem}%[SIMO Fading Number with Feedback]
  \label{thm:feedbackfadingnumber}
 The fading number is 
  \begin{IEEEeqnarray*}{rCl}
    \chi_{\textnormal{FB}}(\{\vect{H}_k\}) & = & \chi(\{\vect{H}_k\})
    \\
    & = & h_{\lambda}\left( \hvect{H}_0 e^{\ii\Theta_0} \middle|
      \big\{\hvect{H}_{\ell}
      e^{\ii\Theta_{\ell}}\big\}_{\ell=-\infty}^{-1} \right)
    - \log 2  \nonumber\\
    && +\: \nr \E{\log \|\vect{H}_0\|^2} - h\left( \vect{H}_0 \middle|
      \vect{H}_{-\infty}^{-1} \right)
  \end{IEEEeqnarray*}

  \vspace{-2.5mm}
\end{whiteboxtheorem}
\begin{overlay}{1}
   \begin{itemize}
   \item
     {\b{$\const{C}_{\textnormal{FB}}(\mathcal{E})=\const{C}(\mathcal{E})$}} asymptotically
\item Feedback is noisefree
\end{itemize}
\vs{2}
\end{overlay}
\end{slide}
% Feedback suppose to help
% even if feedback is noisefree, two capacity is still the same
% any realistic feedback has this result
% Noiseless is upperbound (even our assumption is unrealistic, but it
% contains realistic situation)

%%%slide10 
\begin{slide}
  \begin{center}
    \textbf{\Large Intuition behind the Result}
  \end{center}
 \vspace{3mm}
 \begin{itemize}
 \item Feedback helps: past $\{\vect{H}_k\}$ known
   \item Problem: $h(\{\vect{H}_k\}) > -\infty$
%even with feedback, still noncoherent
     \item Improve the power allocation \\
For ${\r \beta} > 0$ :
\begin{IEEEeqnarray*}{rCl}
  \IEEEeqnarraymulticol{3}{l}{
    \varlimsup_{\mathcal{E}\uparrow\infty} \{\log\log{{\r\beta}}\mathcal{E} - \log\log\mathcal{E} \}
  }\nonumber\\\quad
  & = & \varlimsup_{\mathcal{E}\uparrow\infty} \{ \log(\log{\r{\beta}} + \log\mathcal{E}) -
  \log\log\mathcal{E} \}
  \\
  & = & \varlimsup_{\mathcal{E}\uparrow\infty} \{ \log(\log\mathcal{E}) -
  \log\log\mathcal{E} \} 
  \\
  & = & 0 
\end{IEEEeqnarray*}
\end{itemize}
\vs{2}
\end{slide}


%%%slide10 
\begin{slide}
  \begin{center}
    \textbf{\Large Proof: Challenges}
  \end{center}
 \vspace{3mm}
   \begin{itemize}
     \item $\{\vect{H}_k\}$: very general
   \item $\because$ Feedback $\Rightarrow$ input, fading, and noise are
     dependent
   \item $\because$ Feedback $\Rightarrow$ input is nonstationary
   \end{itemize}
\vs{2}
\end{slide}
% Before, we proved that stationary input can achieve capacity
%% Proof is long, only show some concepts


%%%slide10 
\begin{slide}
  \begin{center}
    \textbf{\Large Proof: Outline}
  \end{center}
 \vspace{3mm}

  \begin{IEEEeqnarray*}{rCl}
 \const{C}=\lim_{n\to \infty} \frac{1}{n} \sup_{Q\in \set{P}(\set{X}^n)} I(X_1^n;Y_1^n)   
  \end{IEEEeqnarray*}
\begin{itemize}
 \item $\const{C}(\mathcal{E})\le\const{C}_{{\b\textnormal{FB}}}(\mathcal{E})$  
 \end{itemize}


\vs{2}
\end{slide}


%%%slide10 
\begin{slide}
  \begin{center}
    \textbf{\Large Proof: Outline (Cont')}
  \end{center}
 \vspace{3mm}

\begin{itemize}
\item By Fano's
\begin{IEEEeqnarray*}{rCl}
  \const{R}_{\textnormal{FB}}(\mathcal{E}) 
  & \le & {\b \frac{1}{n} I\big(M;\vect{Y}_1^n\big) }
  + \frac{\log 2}{n} + P_{\textnormal{e}}^{(n)}\const{R}_{\textnormal{FB}}(\mathcal{E}) 
  + \frac{\eps_n}{n}
\end{IEEEeqnarray*} 
where $P_{\textnormal{e}}^{(n)}$: error probability
\item 
 \begin{IEEEeqnarray*}{rCl}
\frac{1}{n} I\big(M;\vect{Y}_1^n\big)
&=& \frac{1}{n}\sum_{k=1}^n I\big(M; \vect{Y}_k \big|
  \vect{Y}_1^{k-1} \big)\\
& = &\frac{1}{n}\sum_{k={\r{1}}}^{{\r{\kappa}}} I\big(M; \vect{Y}_k
  \big|\vect{Y}_1^{k-1} \big) 
  + \frac{1}{n}\sum_{k={\r{\kappa+1}}}^{{\r{n}}} {\b{I\big(M;
  \vect{Y}_k \big|\vect{Y}_1^{k-1} \big)}}    
  \end{IEEEeqnarray*}
\end{itemize}

\vs{2}
\end{slide}
% we only need to find upper bound because lower bound is just
% capacity without feedback
% average power constraint is upper bound, so if average is proved, peak is
% also proved
% transient phase in Markov (have certain H)  v.s. kind of steady phase 
% Lower bound is equal to C_IID, we do not need to find it.

%%%slide10 
\begin{slide}
  \begin{center}
    \textbf{\Large Proof: Outline (Cont')}
  \end{center}
 \vspace{3mm}
\begin{IEEEeqnarray*}{rCl}
 % \IEEEeqnarraymulticol{3}{l}{
    {\b{I\big(M;\vect{Y}_k \big|\vect{Y}_1^{k-1} \big)}}
%  }\nonumber\\\quad
  & \le &
  I(X_k; \vect{Y}_k) + I\big(\vect{H}_1^{k-1}; \vect{Y}_k \big|
  X_k \big)  - I\big(\vect{Y}_1^{k-1}; \vect{Y}_k \big)  
\end{IEEEeqnarray*}
\begin{overlay}{1}
\begin{itemize}
\item How to compute $I(\vect{H}_1^{k-1}; \vect{Y}_k | X_k, \vect{H}_k)$?
\end{itemize}
\end{overlay}
\vs{2}
\end{slide}
% three terms remain by using deleting, conditioning, chain rule, some
% normal tricks
% 1. memoryless 2. memory 3. correction about H
% if there is no feedback, this term is zero for certain. but now we
% have feedback, we are not certain whether I( | X_k, H_k)
% is zero?
% I( | X_k, H_k) is the term being deleted.


%%%slide11 
\begin{slide}\addtocounter{slide}{-1}
\k
  \begin{center}
    \textbf{\Large Causal Interpretation}
  \end{center}
 \vspace{3mm}
\begin{itemize}
\item Ordered list:  $(M,H_1,...,H_k,Z_1,...,Z_k,F_1,X_1,Y_1,F_2,X_2,Y_2,...,F_k,X_k,Y_k)$
\g
\item
  $(M,H_1,...,H_k,Z_1,...,Z_k,F_1,X_1,Y_1,F_2,X_2,Y_2,...,F_k,X_k,Y_k)$
\item
  $(M,H_1,...,H_k,Z_1,...,Z_k,F_1,X_1,Y_1,F_2,X_2,Y_2,...,F_k,X_k,Y_k)$
\item $(M,H_1,...,H_k,Z_1,...,Z_k,F_1,X_1,Y_1,F_2,X_2,Y_2,...,F_k,X_k,Y_k)$
\end{itemize}
\vs{2}
\end{slide}
% explain the list
% right order
% ordered list is "casaully created (generated)!"

%%%slide11 
\begin{slide}\addtocounter{slide}{-1}
\k
  \begin{center}
    \textbf{\Large Causal Interpretation}
  \end{center}
 \vspace{3mm}
\begin{itemize}
\g
\item Ordered list:  $(M,H_1,...,H_k,Z_1,...,Z_k,F_1,X_1,Y_1,F_2,X_2,Y_2,...,F_k,X_k,Y_k)$
\k
\item
  $({\r M,H_1,...,H_k,Z_1,...,Z_k,}F_1,X_1,Y_1,F_2,X_2,Y_2,...,F_k,X_k,Y_k)$
\g
\item
  $(M,H_1,...,H_k,Z_1,...,Z_k,F_1,X_1,Y_1,F_2,X_2,Y_2,...,F_k,X_k,Y_k)$
\item $(M,H_1,...,H_k,Z_1,...,Z_k,F_1,X_1,Y_1,F_2,X_2,Y_2,...,F_k,X_k,Y_k)$
\end{itemize}
\vs{2}
\end{slide}

%%%slide11 
\begin{slide}\addtocounter{slide}{-1}
\k
  \begin{center}
    \textbf{\Large Causal Interpretation}
  \end{center}
 \vspace{3mm}
\begin{itemize}
\g
\item Ordered list:  $(M,H_1,...,H_k,Z_1,...,Z_k,F_1,X_1,Y_1,F_2,X_2,Y_2,...,F_k,X_k,Y_k)$
\item
  $(M,H_1,...,H_k,Z_1,...,Z_k,F_1,X_1,Y_1,F_2,X_2,Y_2,...,F_k,X_k,Y_k)$
\k
\item
  $(M,H_1,...,H_k,Z_1,...,Z_k,{\r F_1,X_1,Y_1,}F_2,X_2,Y_2,...,F_k,X_k,Y_k)$
\g
\item $(M,H_1,...,H_k,Z_1,...,Z_k,F_1,X_1,Y_1,F_2,X_2,Y_2,...,F_k,X_k,Y_k)$
\end{itemize}
\vs{2}
\end{slide}

%%%slide11 
\begin{slide}\addtocounter{slide}{-1}
\k
  \begin{center}
    \textbf{\Large Causal Interpretation}
  \end{center}
 \vspace{3mm}
\begin{itemize}
\g
\item Ordered list:  $(M,H_1,...,H_k,Z_1,...,Z_k,F_1,X_1,Y_1,F_2,X_2,Y_2,...,F_k,X_k,Y_k)$

\item
  $(M,H_1,...,H_k,Z_1,...,Z_k,F_1,X_1,Y_1,F_2,X_2,Y_2,...,F_k,X_k,Y_k)$
\item
  $(M,H_1,...,H_k,Z_1,...,Z_k,F_1,X_1,Y_1,F_2,X_2,Y_2,...,F_k,X_k,Y_k)$
\k
\item $(M,H_1,...,H_k,Z_1,...,Z_k,F_1,X_1,Y_1,{\r F_2,X_2,Y_2,}...,F_k,X_k,Y_k)$
\end{itemize}
\vs{2}
\end{slide}



%%%slide11 
\begin{slide}
  \begin{center}
    \textbf{\Large Causal Interpretation}
  \end{center}
 \vspace{3mm}
 %\begin{itemize}
$({\r M,H_1,...,H_k,Z_1,...,Z_k,}F_1,X_1,Y_1,F_2,X_2,Y_2,...,F_k,X_k,Y_k)$
%\end{itemize}
%\begin{overlay}{1}
\begin{figure*}[t]
  \centering
  \psfrag{X1}{$X_1$}
\psfrag{X2}{$X_2$}
\psfrag{X3}{$X_3$}
\psfrag{Xk1}{\scriptsize{$X_{k-1}$}}
\psfrag{Xk}{$X_k$}

\psfrag{F1}{$\vect{F}_1$}
\psfrag{F2}{$\vect{F}_2$}
\psfrag{F3}{$\vect{F}_3$}
\psfrag{Fk1}{\scriptsize{$\vect{F}_{k-1}$}}
\psfrag{Fk}{$\vect{F}_k$}

\psfrag{Z1}{$\vect{Z}_1$}
\psfrag{Z2}{$\vect{Z}_2$}
\psfrag{Z3}{$\vect{Z}_3$}
\psfrag{Zk1}{$\vect{Z}_{k-1}$}
\psfrag{Zk}{$\vect{Z}_k$}

\psfrag{Y1}{$\vect{Y}_1$}
\psfrag{Y2}{$\vect{Y}_2$}
\psfrag{Y3}{$\vect{Y}_3$}
\psfrag{Yk1}{\scriptsize{$\vect{Y}_{k-1}$}}
\psfrag{Yk}{$\vect{Y}_k$}

\psfrag{H1}{$\vect{H}_1$}
\psfrag{H2}{$\vect{H}_2$}
\psfrag{H3}{$\vect{H}_3$}
\psfrag{Hk1}{$\vect{H}_{k-1}$}
\psfrag{Hk}{$\vect{H}_k$}

\psfrag{M}{$M$}
  \includegraphics[width=1.0\textwidth]{V3.eps}
  \caption{The causality graph}
  \label{fig:fig2}
\end{figure*}
\vs{2}
%\end{overlay}
\end{slide}



%%%slide11 
\begin{slide}
  \begin{center}
    \textbf{\Large Causal Interpretation}
  \end{center}
 \vspace{3mm}
 %\begin{itemize}
 $(M,H_1,...,H_k,Z_1,...,Z_k,{\r F_1,X_1,Y_1,}F_2,X_2,Y_2,...,F_k,X_k,Y_k)$
%\end{itemize}
%\begin{overlay}{1}
\begin{figure*}[t]
  \centering
  \psfrag{X1}{$X_1$}
\psfrag{X2}{$X_2$}
\psfrag{X3}{$X_3$}
\psfrag{Xk1}{\scriptsize{$X_{k-1}$}}
\psfrag{Xk}{$X_k$}

\psfrag{F1}{$\vect{F}_1$}
\psfrag{F2}{$\vect{F}_2$}
\psfrag{F3}{$\vect{F}_3$}
\psfrag{Fk1}{\scriptsize{$\vect{F}_{k-1}$}}
\psfrag{Fk}{$\vect{F}_k$}

\psfrag{Z1}{$\vect{Z}_1$}
\psfrag{Z2}{$\vect{Z}_2$}
\psfrag{Z3}{$\vect{Z}_3$}
\psfrag{Zk1}{$\vect{Z}_{k-1}$}
\psfrag{Zk}{$\vect{Z}_k$}

\psfrag{Y1}{$\vect{Y}_1$}
\psfrag{Y2}{$\vect{Y}_2$}
\psfrag{Y3}{$\vect{Y}_3$}
\psfrag{Yk1}{\scriptsize{$\vect{Y}_{k-1}$}}
\psfrag{Yk}{$\vect{Y}_k$}

\psfrag{H1}{$\vect{H}_1$}
\psfrag{H2}{$\vect{H}_2$}
\psfrag{H3}{$\vect{H}_3$}
\psfrag{Hk1}{$\vect{H}_{k-1}$}
\psfrag{Hk}{$\vect{H}_k$}

\psfrag{M}{$M$}
  \includegraphics[width=1.0\textwidth]{V2.eps}
  \caption{The causality graph}
  \label{fig:fig2}
\end{figure*}
\vs{2}
%\end{overlay}
\end{slide}


%%%slide11 
\begin{slide}
  \begin{center}
    \textbf{\Large Causal Interpretation}
  \end{center}
 \vspace{3mm}
 %\begin{itemize}
$(M,H_1,...,H_k,Z_1,...,Z_k,F_1,X_1,Y_1,{\r F_2,X_2,Y_2,}...,F_k,X_k,Y_k)$
%\end{itemize}
%\begin{overlay}{1}
\begin{figure*}[t]
  \centering
  \psfrag{X1}{$X_1$}
\psfrag{X2}{$X_2$}
\psfrag{X3}{$X_3$}
\psfrag{Xk1}{\scriptsize{$X_{k-1}$}}
\psfrag{Xk}{$X_k$}

\psfrag{F1}{$\vect{F}_1$}
\psfrag{F2}{$\vect{F}_2$}
\psfrag{F3}{$\vect{F}_3$}
\psfrag{Fk1}{\scriptsize{$\vect{F}_{k-1}$}}
\psfrag{Fk}{$\vect{F}_k$}

\psfrag{Z1}{$\vect{Z}_1$}
\psfrag{Z2}{$\vect{Z}_2$}
\psfrag{Z3}{$\vect{Z}_3$}
\psfrag{Zk1}{$\vect{Z}_{k-1}$}
\psfrag{Zk}{$\vect{Z}_k$}

\psfrag{Y1}{$\vect{Y}_1$}
\psfrag{Y2}{$\vect{Y}_2$}
\psfrag{Y3}{$\vect{Y}_3$}
\psfrag{Yk1}{\scriptsize{$\vect{Y}_{k-1}$}}
\psfrag{Yk}{$\vect{Y}_k$}

\psfrag{H1}{$\vect{H}_1$}
\psfrag{H2}{$\vect{H}_2$}
\psfrag{H3}{$\vect{H}_3$}
\psfrag{Hk1}{$\vect{H}_{k-1}$}
\psfrag{Hk}{$\vect{H}_k$}

\psfrag{M}{$M$}
  \includegraphics[width=1.0\textwidth]{V1.eps}
  \caption{The causality graph}
  \label{fig:fig2}
\end{figure*}
\vs{2}
%\end{overlay}
\end{slide}


%%%slide11 
\begin{slide}
  \begin{center}
    \textbf{\Large Causal Interpretation}
  \end{center}
 \vspace{3mm}
% \begin{itemize}
$(M,H_1,...,H_k,Z_1,...,Z_k,F_1,X_1,Y_1,F_2,X_2,Y_2,...,F_k,X_k,Y_k)$
%\end{itemize}
%\begin{overlay}{1}
\begin{figure*}[t]
  \centering
  \psfrag{X1}{$X_1$}
\psfrag{X2}{$X_2$}
\psfrag{X3}{$X_3$}
\psfrag{Xk1}{\scriptsize{$X_{k-1}$}}
\psfrag{Xk}{$X_k$}

\psfrag{F1}{$\vect{F}_1$}
\psfrag{F2}{$\vect{F}_2$}
\psfrag{F3}{$\vect{F}_3$}
\psfrag{Fk1}{\scriptsize{$\vect{F}_{k-1}$}}
\psfrag{Fk}{$\vect{F}_k$}

\psfrag{Z1}{$\vect{Z}_1$}
\psfrag{Z2}{$\vect{Z}_2$}
\psfrag{Z3}{$\vect{Z}_3$}
\psfrag{Zk1}{$\vect{Z}_{k-1}$}
\psfrag{Zk}{$\vect{Z}_k$}

\psfrag{Y1}{$\vect{Y}_1$}
\psfrag{Y2}{$\vect{Y}_2$}
\psfrag{Y3}{$\vect{Y}_3$}
\psfrag{Yk1}{\scriptsize{$\vect{Y}_{k-1}$}}
\psfrag{Yk}{$\vect{Y}_k$}

\psfrag{H1}{$\vect{H}_1$}
\psfrag{H2}{$\vect{H}_2$}
\psfrag{H3}{$\vect{H}_3$}
\psfrag{Hk1}{$\vect{H}_{k-1}$}
\psfrag{Hk}{$\vect{H}_k$}

\psfrag{M}{$M$}
  \includegraphics[width=1.0\textwidth]{V.eps}
  \caption{The causality graph of our model}
  \label{fig:fig2}
\end{figure*}
\vs{2}
%\end{overlay}
\end{slide}



%%%slide12
\begin{slide}
  \begin{center}
    \textbf{\Large Causal Interpretation: Example}
  \end{center}
 \vspace{3mm}

How to compute $I(\vect{H}_1^{k-1}; \vect{Y}_k | X_k, \vect{H}_k)$?
\begin{overlay}{1}
\begin{figure*}[t]
  \centering
     \psfrag{X1}{$X_1$}
\psfrag{X2}{$X_2$}
\psfrag{X3}{$X_3$}
\psfrag{Xk1}{$X_{k-1}$}
\psfrag{Xk}{${\r{X_k}}$}

\psfrag{F1}{$\vect{F}_1$}
\psfrag{F2}{$\vect{F}_2$}
\psfrag{F3}{$\vect{F}_3$}
\psfrag{Fk1}{$\vect{F}_{k-1}$}
\psfrag{Fk}{$\vect{F}_k$}

\psfrag{Z1}{$\vect{Z}_1$}
\psfrag{Z2}{$\vect{Z}_2$}
\psfrag{Z3}{$\vect{Z}_3$}
\psfrag{Zk1}{$\vect{Z}_{k-1}$}
\psfrag{Zk}{$\vect{Z}_k$}

\psfrag{Y1}{$\vect{Y}_1$}
\psfrag{Y2}{$\vect{Y}_2$}
\psfrag{Y3}{$\vect{Y}_3$}
\psfrag{Yk1}{$\vect{Y}_{k-1}$}
\psfrag{Yk}{${\b{\vect{Y}_k}}$}

\psfrag{H1}{${\b{\vect{H}_1}}$}
\psfrag{H2}{${\b{\vect{H}_2}}$}
\psfrag{H3}{${\b{\vect{H}_3}}$}
\psfrag{Hk1}{${\b{\vect{H}_{k-1}}}$}
\psfrag{Hk}{${\r{\vect{H}_k}}$}

\psfrag{M}{$M$}
  \includegraphics[width=1.0\textwidth]{HY.eps}
  \caption{${\b{\vect{H}_1^{k-1}}} \indep {\b{\vect{Y}_k}}$ when
   conditioned on $({\r{X_k}},{\r{\vect{H}_k}})$}
  \label{fig:fig5}
\end{figure*}
\end{overlay}

% \begin{figure}[htbp]
%   \centering
% \psfrag{X1}{$X_1$}
% \psfrag{X2}{$X_2$}
% \psfrag{X3}{$X_3$}
% \psfrag{X4}{$X_4$}
% \psfrag{X5}{$X_5$}
% \psfrag{X6}{$X_6$}
%    \includegraphics[width=0.3\textwidth]{f1.eps}
%   \caption{Toy example}
%   \label{fig:1}
% \end{figure}

% \begin{figure}[htbp]
%   \centering
% \psfrag{X1}{$X_1$}
% \psfrag{X2}{$X_2$}
% \psfrag{X3}{$X_3$}
% \psfrag{X4}{$X_4$}
% \psfrag{X5}{$X_5$}
%    \includegraphics[width=0.2\textwidth]{f3.eps}
%   \caption{$X_4 \indep X_5$ if conditioned on $X_2$}
%   \label{fig:1}
% \end{figure}

\vs{2}
\end{slide}

%%%slide17 
\begin{slide}
  \begin{center}
    \textbf{\Large Proof: Escape to Infinity}
  \end{center}
 \vspace{3mm}
\begin{IEEEeqnarray*}{rCl}
 % \IEEEeqnarraymulticol{3}{l}{
    {\b{I\big(M;\vect{Y}_k \big|\vect{Y}_1^{k-1} \big)}}
%  }\nonumber\quad
  & \le &
  I(X_k; \vect{Y}_k) + I\big(\vect{H}_1^{k-1}; \vect{Y}_k \big|
  X_k \big)  - I\big(\vect{Y}_1^{k-1}; \vect{Y}_k \big)  
\end{IEEEeqnarray*}
\begin{itemize}
\item Case 1: $ |X_k| \ge \xi_{\textnormal{min}}$ \\
Case 2: $ |X_k| \le \xi_{\textnormal{min}}$
\end{itemize}
\vs{2}
\end{slide}
% without feedback, with feedback, it turns out it is true as well
% why this two cases? escape to infinity business; \beta=1
% we implicitly proved that Pr(X_k > \xi_min)=1


%%%slide18 
\begin{slide}
  \begin{center}
    \textbf{\Large Proof: $\E{\|\vect{H}_k\|^2|X_k|^2}$}
  \end{center}
 \vspace{3mm}
 \begin{itemize}
 \item $\E{\|\vect{H}_k\|^2|X_k|^2}=\E{\|\vect{H}_k\|^2}\E{|X_k|^2}$\
   \item Case 1: $\|\vect{H}_k\|^2 \ge t$\\
Case 2:  $\|\vect{H}_k\|^2 < t$
\item By Markov inequality:
\begin{equation*}
  \label{eq:21}
  {\b{\Prv{\|\vect{H}_k\|^2\ge t}\le\frac{\E{\|\vect{H}_k\|^2}}{t}}}
\end{equation*}
\end{itemize}
\vs{2}
\end{slide}
% A is independent of B
% be independent "of" each other
% choose t depend on E, as E goes to infinity, the Pr{>t} will be
% bounded by zero, so case one disapear.

%%%slide13
\begin{slide}
  \begin{center}
    \textbf{\Large Summary}
  \end{center}
\vspace{3mm}
\begin{itemize}
\item
  ${\b{\const{C}_{\textnormal{FB}}(\mathcal{E})}}=\const{C}(\mathcal{E})$
  asymptotically
% lower bound and upper bound
\item $\{\vect{H}_k\}$ and feedback are general 
\item $\because$ Feedback is noiseless $\Rightarrow$ the best we can get
%Because $C_{FB}=C$ and we assume feedback is noiseless, it is the best
%  that we can get in high SNR
\item How about MISO?
\end{itemize} 
\vs{2}
\end{slide}
% MISO: not only mag, phase (circularly), but optimal direction?

%%%slide14
\begin{slide}
  \begin{center}
    \textbf{\Large Thank you :-)}
  \end{center}
\end{slide}

\end{document}
